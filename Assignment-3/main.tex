\documentclass[journal,12pt,twocolumn]{IEEEtran}

\usepackage{setspace}
\usepackage{gensymb}
\singlespacing
\usepackage[cmex10]{amsmath}

\usepackage{amsthm}
\usepackage{amsmath,amssymb}
\usepackage{mathrsfs}
\usepackage{txfonts}
\usepackage{stfloats}
\usepackage{bm}
\usepackage{cite}
\usepackage{cases}
\usepackage{subfig}

\usepackage{longtable}
\usepackage{multirow}
\usepackage{float}
\usepackage{enumitem}
\usepackage{mathtools}
\usepackage{steinmetz}
\usepackage{tikz}
\usepackage{circuitikz}
\usepackage{verbatim}
%\usepackage{tfrupee}
\usepackage[breaklinks=true]{hyperref}
\usepackage{graphicx}
\usepackage{tkz-euclide}
\usepackage{stackengine}
\usetikzlibrary{calc,math}
\usepackage{listings}
    \usepackage{color}                                            %%
    \usepackage{array}                                            %%
    \usepackage{longtable}                                        %%
    \usepackage{calc}                                             %%
    \usepackage{multirow}                                         %%
    \usepackage{hhline}                                           %%
    \usepackage{ifthen}                                           %%
    \usepackage{lscape}     
\usepackage{multicol}
\usepackage{chngcntr}
\usepackage{bm}
\DeclareMathOperator*{\Res}{Res}

\renewcommand\thesection{\arabic{section}}
\renewcommand\thesubsection{\thesection.\arabic{subsection}}
\renewcommand\thesubsubsection{\thesubsection.\arabic{subsubsection}}

\renewcommand\thesectiondis{\arabic{section}}
\renewcommand\thesubsectiondis{\thesectiondis.\arabic{subsection}}
\renewcommand\thesubsubsectiondis{\thesubsectiondis.\arabic{subsubsection}}
\newtheorem{lemma}{Lemma}
\newtheorem{sublemma}{Lemma}[lemma]

\hyphenation{op-tical net-works semi-conduc-tor}
\def\inputGnumericTable{}                                 %%

\lstset{
%language=C,
frame=single, 
breaklines=true,
columns=fullflexible
}
\makeatletter
\setlength{\@fptop}{0pt}
\makeatother
\begin{document}

\newcommand{\BEQA}{\begin{eqnarray}}
\newcommand{\EEQA}{\end{eqnarray}}
\newcommand{\define}{\stackrel{\triangle}{=}}
\bibliographystyle{IEEEtran}
\raggedbottom
\setlength{\parindent}{0pt}
\providecommand{\mbf}{\mathbf}
\providecommand{\pr}[1]{\ensuremath{\Pr\left(#1\right)}}
\providecommand{\qfunc}[1]{\ensuremath{Q\left(#1\right)}}
\providecommand{\sbrak}[1]{\ensuremath{{}\left[#1\right]}}
\providecommand{\lsbrak}[1]{\ensuremath{{}\left[#1\right.}}
\providecommand{\rsbrak}[1]{\ensuremath{{}\left.#1\right]}}
\providecommand{\brak}[1]{\ensuremath{\left(#1\right)}}
\providecommand{\lbrak}[1]{\ensuremath{\left(#1\right.}}
\providecommand{\rbrak}[1]{\ensuremath{\left.#1\right)}}
\providecommand{\cbrak}[1]{\ensuremath{\left\{#1\right\}}}
\providecommand{\lcbrak}[1]{\ensuremath{\left\{#1\right.}}
\providecommand{\rcbrak}[1]{\ensuremath{\left.#1\right\}}}
\DeclarePairedDelimiter\ceil{\lceil}{\rceil}
\DeclarePairedDelimiter\floor{\lfloor}{\rfloor}
\theoremstyle{remark}
\newtheorem{rem}{Remark}
\newcommand{\sgn}{\mathop{\mathrm{sgn}}}
\providecommand{\abs}[1]{\vert#1\vert}
\providecommand{\res}[1]{\Res\displaylimits_{#1}} 
\providecommand{\norm}[1]{\lVert#1\rVert}
%\providecommand{\norm}[1]{\lVert#1\rVert}
\providecommand{\mtx}[1]{\mathbf{#1}}
\providecommand{\mean}[1]{E[ #1 ]}
\providecommand{\fourier}{\overset{\mathcal{F}}{ \rightleftharpoons}}
%\providecommand{\hilbert}{\overset{\mathcal{H}}{ \rightleftharpoons}}
\providecommand{\system}{\overset{\mathcal{H}}{ \longleftrightarrow}}
	%\newcommand{\solution}[2]{\textbf{Solution:}{#1}}
\newcommand{\solution}{\noindent \textbf{Solution: }}
\newcommand{\cosec}{\,\text{cosec}\,}
\newcommand*{\permcomb}[4][0mu]{{{}^{#3}\mkern#1#2_{#4}}}
\newcommand*{\perm}[1][-3mu]{\permcomb[#1]{P}}
\newcommand*{\comb}[1][-1mu]{\permcomb[#1]{C}}
\newcommand\xrowht[2][0]{\addstackgap[.5\dimexpr#2\relax]{\vphantom{#1}}}
\providecommand{\dec}[2]{\ensuremath{\overset{#1}{\underset{#2}{\gtrless}}}}
\newcommand{\myvec}[1]{\ensuremath{\begin{pmatrix}#1\end{pmatrix}}}
\newcommand{\mydet}[1]{\ensuremath{\begin{vmatrix}#1\end{vmatrix}}}
\numberwithin{equation}{subsection}
\makeatletter
\@addtoreset{figure}{problem}
\makeatother
\let\StandardTheFigure\thefigure
\let\vec\mathbf
\renewcommand{\thefigure}{\theproblem}
\def\putbox#1#2#3{\makebox[0in][l]{\makebox[#1][l]{}\raisebox{\baselineskip}[0in][0in]{\raisebox{#2}[0in][0in]{#3}}}}
     \def\rightbox#1{\makebox[0in][r]{#1}}
     \def\centbox#1{\makebox[0in]{#1}}
     \def\topbox#1{\raisebox{-\baselineskip}[0in][0in]{#1}}
     \def\midbox#1{\raisebox{-0.5\baselineskip}[0in][0in]{#1}}
\vspace{3cm}
\title{Assignment 3}
\author{Chirag Mehta - AI20BTECH11006}
\maketitle
\newpage
\bigskip
\renewcommand{\thefigure}{\theenumi}
\renewcommand{\thetable}{\theenumi}
Download all the python codes from
\begin{lstlisting}
https://github.com/cmaspi/EE3900/tree/main/Assignment-3/code
\end{lstlisting}
latex-tikz codes from 
\begin{lstlisting}
https://github.com/cmaspi/EE3900/blob/main/Assignment-3/main.tex
\end{lstlisting}
\section{Problem}
(Construction Q 2.14) Draw a circle of radius 3 units.
Take two points $\vec{P}$ and $\vec{Q}$ on one of its extended 
diameter each at a distance of $7$ units from its centre.
Draw tangents to the circle from these two points $\vec{P}$ and $\vec{Q}$ 
\section{Solution}

The given parameters are listed in Table \ref{tab:table1}
%
\begin{table}[!ht]
\begin{center}
\begin{tabular}{ | m{2cm} | m{2cm} |} 
\hline
 & Circle \\
\hline
Centre  & $\vec{O}$=\myvec{0\\0} \\ 
\hline
Radius & $r$=3  \\ 
\hline
Radius & $d$=7  \\ 
\hline
\end{tabular}
\end{center}
\caption{Input values}
\label{tab:table1}
\end{table}
%
\begin{lemma}
  \label{lemma/linman/circ/contact/final}
  The points of contact for the tangent drawn from a point 
%
\begin{align}
  \vec{P} = d\vec{e}_1, \text{ where } \vec{e}_1 = \myvec{1\\0}
  \end{align}
  %
  to the circle are given by 
  \begin{align}
    \vec{x} = \frac{r^2}{d}\vec{e}_1  \pm r\sqrt{1 - \frac{r^2}{d^2}} \vec{e}_2
    \label{linman/circ/contact/final}
   \end{align}
%   
\end{lemma}
If $\vec{x}$ be a point of contact for the tangent from $\vec{P}$, 
\begin{align}
PR &\perp RO
\\
 \implies (\vec{O}-\vec{x})^{\top} (\vec{x}-\vec{P}) &= 0
 \\
 \text{or, }  \vec{P}^{\top} \vec{x} &=\norm{\vec{x}}^2 = r^2
 \\
 \implies \vec{e}_1^{\top} \vec{x} &= \frac{r^2}{d}
  \end{align}
  $\because \vec{O} = 0$.  The above equation can be expressed in parametric form as 
 \begin{align}
  \vec{x} = \frac{r^2}{d}\vec{e}_1 + \lambda \vec{e}_2
  \label{linman/circ/contact}
 \end{align}
 Substituting the above in 
 \begin{align}
  \norm{\vec{x}}^2 = r^2,
 \end{align}
 yields
\begin{align}
\norm{\frac{r^2}{d}\vec{e}_1 + \lambda \vec{e}_2}^2&=r^2
\\
\implies \lambda^2 &= r^2\sbrak{1 - \frac{r^2}{d^2}}
\\
\text{or, }\lambda &= \pm r\sqrt{1 - \frac{r^2}{d^2}}
\end{align}
%
Substituting $\lambda $ in \eqref{linman/circ/contact} yields \eqref{linman/circ/contact/final}.  Fig.  \ref{fig:Tangent lines to circle of radius 3 units.} shows all possible tangents
and their points of contact after substituting the numerical values in \eqref{linman/circ/contact/final}.
%

\item Draw a  pair of tangents to a circle of radius 5 units  which are inclined to each other at an angle of $60\degree$.
\\
\solution  The angle between the tangents from $\vec{P}$ is given by 
\begin{lemma}
  Given a circle of radius $r$ and angle $\theta$ between the tangents, the intersection of the tangents and points of contact are
  given by Lemma   \ref{lemma/linman/circ/contact/final}  where 
  \begin{align}
    \implies d &= r\sin \frac{\theta}{2}
  \end{align}
%  
\end{lemma}
\begin{proof}
  From Fig.  \ref{fig:Tangent lines to circle of radius 3 units.},
\begin{align}
  \sin \frac{\theta}{2} &= \frac{r}{d}
  \\
  \implies d &= r\sin \frac{\theta}{2}
\end{align}
\end{proof}
Substituting numerical values and plotting, we obtain Fig. \ref{fig:Tangent lines to circle of radius 5 units.}.


A plot for the planes is given below
% \begin{figure}[!ht]
%     \centering
%     \includegraphics[width=\columnwidth]{plot/figure_1}
%     \caption{Plot of the planes}
%     \label{plot}
% \end{figure}
\end{document}