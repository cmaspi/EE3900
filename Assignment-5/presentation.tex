\documentclass{beamer}
\usepackage{listings}
\lstset{
%language=C,
frame=single, 
breaklines=true,
columns=fullflexible
}
\usepackage{blkarray}
\usepackage{subcaption}
\usepackage{url}
\usepackage{tikz}
\usepackage{tkz-euclide} % loads  TikZ and tkz-base
%\usetkzobj{all}
\usetikzlibrary{calc,math}
\usepackage{float}
\usepackage{bm}
\newcommand{\myvec}[1]{\ensuremath{\begin{pmatrix}#1\end{pmatrix}}}
\providecommand{\brak}[1]{\ensuremath{\left(#1\right)}}
\newcommand\norm[1]{\left\lVert#1\right\rVert}
\renewcommand{\vec}[1]{\boldsymbol{#1}}
\usepackage[export]{adjustbox}
\usepackage[utf8]{inputenc}
\usepackage{amsmath}
\usepackage{physics}
\usepackage{tikz}
\usetikzlibrary{automata, positioning}
\usetheme{Boadilla}
% \providecommand{\pr}[1]{\ensuremath{\Pr\left(#1\right)}}

\title{Assignment 5 Presentation}
\author{Chirag Mehta}
\date{AI20BTECH11006}
\begin{document}

\begin{frame}
\titlepage
\end{frame}

\begin{frame}
\frametitle{Question}
\begin{block}{Quadratic forms Q2.66}
Find the point at which the line 
$\myvec{-1 & 1}\vec{x}=1$ is a tangent to the curve $y^2=4x$    
\end{block}
\end{frame}

\begin{frame}
\frametitle{Theorem}
\begin{theorem}
    The solution to a under-determined system of equations $\vec{Ax}=\vec{b}$ is given by
    \begin{align}
        \vec{x}=\vec{A}^+\vec{b}+(\vec{I}-\vec{A}^+\vec{A})\vec{w} \label{eq:generalisedInverse}
    \end{align}
    where $\vec{A^+}$ is the pseudoinverse of the matrix $\vec{A}$\\
    \end{theorem}
\end{frame}

\begin{frame}
    \begin{proof}
        Let $\vec{Ax}=\vec{b}$ have at least one solution
        \begin{align}
            \vec{Ax}&=\vec{b}\\
            \vec{AA}^+(\vec{Ax})&=\vec{AA}^+(\vec{b})
        \end{align}
        Using property of pseudoinverse
        \begin{align}
            \vec{Ax}=\vec{A}^+\vec{Ab}=\vec{b}
        \end{align}
        Therefore, $\vec{A}^+\vec{b}$ is a specific solution.\\
        The entire set of solution is given by $\vec{A}^+\vec{b}+\vec{k}$, where $\vec{k}$ is a vector in kernel space or null space of $\vec{A}$
        \begin{align}
            \vec{A}\brak{\vec{A}^+\vec{b}+\vec{k}}&=\vec{b}\\
            \vec{b}+\vec{Ak}&=\vec{b}\\
            \vec{Ak}=\vec{0}
        \end{align}
        Any vector in the null space of $\vec{A}$ can be written as
        \begin{align}
            \vec{k}=(\vec{I}-\vec{A}^+\vec{A})\vec{w} \label{eq:nullSpace}
        \end{align}
        where $\vec{w}$ is any vector with appropriate dimension.       
        Now, we will prove that~\eqref{eq:nullSpace} holds true
        \begin{align}
            \vec{k}&=(\vec{I-A}^+\vec{A})\vec{k}\\
            \vec{k}&=\vec{k}+\vec{A}^+(\vec{Ak})\\
            \therefore \vec{k}&=\vec{k}+\vec{A}^+(\vec{0})
        \end{align}
        Therefore any vector in null space of $\vec{A}$ is also in the null space of $ \vec{I-A}^+\vec{A} $
        \begin{align}
            \vec{A}(\vec{I-A}^+\vec{A})\vec{w}=(\vec{A}-\vec{A})\vec{w}=\vec{0}
        \end{align}
        Therefore, the null space of $\vec{A}$ and $\vec{I-A}^+\vec{A}$ are the same
        \begin{align}
            \vec{x}=\vec{A}^+\vec{b}+(\vec{I-A}^+\vec{A})\vec{w}
        \end{align}
        \end{proof}
\end{frame}

\begin{frame}
    \begin{lemma}
        \label{lemma:pseudoinverse}
            For an idempotent matrix the pseudoinverse is the matrix itself
        \end{lemma}
        \begin{proof}
            For an symmetric idempotent matrix
            \begin{align}
                \vec{A}^n=\vec{A}\\
                \vec{AAA}=\vec{A}
            \end{align}
            Therefore, $\vec{A}=\vec{A}^+$ satisfies all the conditions of pseudoinverse which are listed as follows
        \begin{enumerate}
            \item $\vec{AA}^+\vec{A}=\vec{A}$
            \item $\vec{A}^+\vec{AA}^+=\vec{A}^+$
            \item $\vec{AA}^+$ is symmetric
            \item $\vec{A}^+\vec{A}$ is symmetric
        \end{enumerate}
        \end{proof}
\end{frame}

\begin{frame}
\frametitle{Solution}
The general form of a conic is given by
\begin{align}
    \vec{x}^T\vec{Vx}+2\vec{u}^T\vec{x}+f=0 \label{eq:genConic}
\end{align}
for $y^2=4x$
\begin{align}
\vec{V}=\myvec{0 & 0 \\ 0 & 1},\ \vec{u}=\myvec{-2 \\0},\, f=0
\end{align}
\end{frame}

\begin{frame}
\frametitle{Solution Contd.}
For any given conic such that $\vec{V}$ is non-invertible, the point of tangency is given by
\begin{align}
    \vec{Vq}&=k\vec{n}-u \\
    \text{where},\, k& = \frac{\vec{p_1}^T\vec{u}}{\vec{p_1}^T\vec{n}},\, \vec{Vp_1}=0
\end{align}
Clearly
\begin{align}
    \vec{p_1}& =\myvec{1\\0} \\
    \implies k & =\frac{-2}{-1}=2
\end{align}
\end{frame}

\begin{frame}
    \frametitle{Solution Contd.}
    Using the obtained values
\begin{align}
    &\vec{Vq}=2\vec{n}-\vec{u}\\
    &\vec{Vq}=\myvec{-2\\2}-\myvec{-2\\0}=2\vec{e_2}
\end{align}
Using \eqref{eq:generalisedInverse} and \eqref{lemma:pseudoinverse}
\begin{align}
    \vec{q}&=2\vec{Ve_2}+(\vec{I-V^2})\vec{w}\\
    \vec{q}&=2\vec{e_2}+\lambda\vec{e_1} \label{eq:q_almostfound}
\end{align}
\end{frame}

\begin{frame}
\frametitle{Solution Contd.}
Using \eqref{eq:q_almostfound} and \eqref{eq:genConic}, we get
\begin{align}
    2\vec{q}^T\vec{e_2}-4\vec{e_1}^T\vec{q}=0
\end{align}
Evaluating this, we get
\begin{align}
    4-\lambda&=0 \\
    \implies \vec{q}&=\myvec{1\\2}
\end{align}
\end{frame}



% \begin{frame}
%     \frametitle{Solution Contd.}
    
% \end{frame}

\begin{frame}
\frametitle{Solution Contd.}
\begin{figure}[!h]
    \centering
    \includegraphics[width=0.8\columnwidth]{plot/figure_1}
    \caption{Plot of the line and parabola}
    \label{plot}
\end{figure}
\end{frame}


\end{document}