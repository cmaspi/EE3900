\documentclass[journal,12pt,twocolumn]{IEEEtran}

\usepackage{setspace}
\usepackage{gensymb}
\singlespacing
\usepackage[cmex10]{amsmath}

\usepackage{amsthm}
\usepackage{amsmath,amssymb}
\usepackage{mathrsfs}
\usepackage{txfonts}
\usepackage{stfloats}
\usepackage{bm}
\usepackage{cite}
\usepackage{cases}
\usepackage{subfig}

\usepackage{longtable}
\usepackage{multirow}
\usepackage{float}
\usepackage{enumitem}
\usepackage{mathtools}
\usepackage{steinmetz}
\usepackage{tikz}
\usepackage{circuitikz}
\usepackage{verbatim}
%\usepackage{tfrupee}
\usepackage[breaklinks=true]{hyperref}
\usepackage{graphicx}
\usepackage{tkz-euclide}
\usepackage{stackengine}
\usetikzlibrary{calc,math}
\usepackage{listings}
    \usepackage{color}                                            %%
    \usepackage{array}                                            %%
    \usepackage{longtable}                                        %%
    \usepackage{calc}                                             %%
    \usepackage{multirow}                                         %%
    \usepackage{hhline}                                           %%
    \usepackage{ifthen}                                           %%
    \usepackage{lscape}     
\usepackage{multicol}
\usepackage{chngcntr}
\usepackage{bm}
\DeclareMathOperator*{\Res}{Res}

\renewcommand\thesection{\arabic{section}}
\renewcommand\thesubsection{\thesection.\arabic{subsection}}
\renewcommand\thesubsubsection{\thesubsection.\arabic{subsubsection}}

\renewcommand\thesectiondis{\arabic{section}}
\renewcommand\thesubsectiondis{\thesectiondis.\arabic{subsection}}
\renewcommand\thesubsubsectiondis{\thesubsectiondis.\arabic{subsubsection}}


\hyphenation{op-tical net-works semi-conduc-tor}
\def\inputGnumericTable{}                                 %%

\lstset{
%language=C,
frame=single, 
breaklines=true,
columns=fullflexible
}
\makeatletter
\setlength{\@fptop}{0pt}
\makeatother
\begin{document}

\newtheorem{theorem}{Theorem}[section]
\newtheorem{problem}{Problem}
\newtheorem{proposition}{Proposition}[section]
\newtheorem{lemma}{Lemma}[section]
\newtheorem{corollary}[theorem]{Corollary}
\newtheorem{example}{Example}[section]
\newtheorem{definition}[problem]{Definition}

\newcommand{\BEQA}{\begin{eqnarray}}
\newcommand{\EEQA}{\end{eqnarray}}
\newcommand{\define}{\stackrel{\triangle}{=}}
\bibliographystyle{IEEEtran}
\raggedbottom
\setlength{\parindent}{0pt}
\providecommand{\mbf}{\mathbf}
\providecommand{\pr}[1]{\ensuremath{\Pr\left(#1\right)}}
\providecommand{\qfunc}[1]{\ensuremath{Q\left(#1\right)}}
\providecommand{\sbrak}[1]{\ensuremath{{}\left[#1\right]}}
\providecommand{\lsbrak}[1]{\ensuremath{{}\left[#1\right.}}
\providecommand{\rsbrak}[1]{\ensuremath{{}\left.#1\right]}}
\providecommand{\brak}[1]{\ensuremath{\left(#1\right)}}
\providecommand{\lbrak}[1]{\ensuremath{\left(#1\right.}}
\providecommand{\rbrak}[1]{\ensuremath{\left.#1\right)}}
\providecommand{\cbrak}[1]{\ensuremath{\left\{#1\right\}}}
\providecommand{\lcbrak}[1]{\ensuremath{\left\{#1\right.}}
\providecommand{\rcbrak}[1]{\ensuremath{\left.#1\right\}}}
\DeclarePairedDelimiter\ceil{\lceil}{\rceil}
\DeclarePairedDelimiter\floor{\lfloor}{\rfloor}
\theoremstyle{remark}
\newtheorem{rem}{Remark}
\newcommand{\sgn}{\mathop{\mathrm{sgn}}}
\providecommand{\abs}[1]{\vert#1\vert}
\providecommand{\res}[1]{\Res\displaylimits_{#1}} 
\providecommand{\norm}[1]{\lVert#1\rVert}
%\providecommand{\norm}[1]{\lVert#1\rVert}
\providecommand{\mtx}[1]{\mathbf{#1}}
\providecommand{\mean}[1]{E[ #1 ]}
\providecommand{\fourier}{\overset{\mathcal{F}}{ \rightleftharpoons}}
%\providecommand{\hilbert}{\overset{\mathcal{H}}{ \rightleftharpoons}}
\providecommand{\system}{\overset{\mathcal{H}}{ \longleftrightarrow}}
	%\newcommand{\solution}[2]{\textbf{Solution:}{#1}}
\newcommand{\solution}{\noindent \textbf{Solution: }}
\newcommand{\cosec}{\,\text{cosec}\,}
\newcommand*{\permcomb}[4][0mu]{{{}^{#3}\mkern#1#2_{#4}}}
\newcommand*{\perm}[1][-3mu]{\permcomb[#1]{P}}
\newcommand*{\comb}[1][-1mu]{\permcomb[#1]{C}}
\newcommand\xrowht[2][0]{\addstackgap[.5\dimexpr#2\relax]{\vphantom{#1}}}
\providecommand{\dec}[2]{\ensuremath{\overset{#1}{\underset{#2}{\gtrless}}}}
\newcommand{\myvec}[1]{\ensuremath{\begin{pmatrix}#1\end{pmatrix}}}
\newcommand{\mydet}[1]{\ensuremath{\begin{vmatrix}#1\end{vmatrix}}}
\numberwithin{equation}{subsection}
\makeatletter
\@addtoreset{figure}{problem}
\makeatother
\let\StandardTheFigure\thefigure
\let\vec\mathbf
\renewcommand{\thefigure}{\theproblem}
\def\putbox#1#2#3{\makebox[0in][l]{\makebox[#1][l]{}\raisebox{\baselineskip}[0in][0in]{\raisebox{#2}[0in][0in]{#3}}}}
     \def\rightbox#1{\makebox[0in][r]{#1}}
     \def\centbox#1{\makebox[0in]{#1}}
     \def\topbox#1{\raisebox{-\baselineskip}[0in][0in]{#1}}
     \def\midbox#1{\raisebox{-0.5\baselineskip}[0in][0in]{#1}}
\vspace{3cm}
\title{Assignment 5}
\author{Chirag Mehta - AI20BTECH11006}
\maketitle
\newpage
\bigskip
\renewcommand{\thefigure}{\theenumi}
\renewcommand{\thetable}{\theenumi}
Download all the python codes from
\begin{lstlisting}
https://github.com/cmaspi/EE3900/tree/main/Assignment-5/code
\end{lstlisting}
latex-tikz codes from 
\begin{lstlisting}
https://github.com/cmaspi/EE3900/blob/main/Assignment-5/main.tex
\end{lstlisting}
\section{Problem}
(Quadratic forms Q2.66) Find the point at which the line 
$\myvec{-1 & 1}\vec{x}=1$ is a tangent to the curve $y^2=4x$
\section{Solution}

\begin{theorem}
The solution to a under-determined system of equations $\vec{Ax}=\vec{b}$ is given by
\begin{align}
    \vec{x}=\vec{A}^+\vec{b}+(\vec{I}-\vec{A}^+\vec{A})\vec{w} \label{eq:generalisedInverse}
\end{align}
where $\vec{A^+}$ is the pseudoinverse of the matrix $\vec{A}$\\
\end{theorem}

\begin{proof}
Let $\vec{Ax}=\vec{b}$ have at least one solution
\begin{align}
    \vec{Ax}&=\vec{b}\\
    \vec{AA}^+(\vec{Ax})&=\vec{AA}^+(\vec{b})
\end{align}
Using property of pseudoinverse
\begin{align}
    \vec{Ax}=\vec{A}^+\vec{Ab}=\vec{b}
\end{align}
Therefore, $\vec{A}^+\vec{b}$ is a specific solution.\\
The entire set of solution is given by $\vec{A}^+\vec{b}+\vec{k}$, where $\vec{k}$ is a vector in kernel space or null space of $\vec{A}$
\begin{align}
    \vec{A}\brak{\vec{A}^+\vec{b}+\vec{k}}&=\vec{b}\\
    \vec{b}+\vec{Ak}&=\vec{b}\\
    \vec{Ak}=\vec{0}
\end{align}
Any vector in the null space of $\vec{A}$ can be written as
\begin{align}
    \vec{k}=(\vec{I}-\vec{A}^+\vec{A})\vec{w} \label{eq:nullSpace}
\end{align}
where $\vec{w}$ is any vector with appropriate dimension.\\
Now, we will prove that \eqref{eq:nullSpace} holds true
\begin{align}
    \vec{k}&=(\vec{I-A}^+\vec{A})\vec{k}\\
    \vec{k}&=\vec{k}+\vec{A}^+(\vec{Ak})\\
    \therefore \vec{k}&=\vec{k}+\vec{A}^+(\vec{0})
\end{align}
Therefore any vector in null space of $\vec{A}$ is also in the null space of $ \vec{I-A}^+\vec{A} $
\begin{align}
    \vec{A}(\vec{I-A}^+\vec{A})\vec{w}=(\vec{A}-\vec{A})\vec{w}=\vec{0}
\end{align}
Therefore, the null space of $\vec{A}$ and $\vec{I-A}^+\vec{A}$ are the same
\begin{align}
    \vec{x}=\vec{A}^+\vec{b}+(\vec{I-A}^+\vec{A})\vec{w}
\end{align}
\end{proof}
\begin{lemma}
\label{lemma:pseudoinverse}
    For an idempotent matrix the pseudoinverse is the matrix itself
\end{lemma}
\begin{proof}
    For an symmetric idempotent matrix
    \begin{align}
        \vec{A}^n=\vec{A}\\
        \vec{AAA}=\vec{A}
    \end{align}
    Therefore, $\vec{A}=\vec{A}^+$ satisfies all the conditions of pseudoinverse which are listed as follows
\begin{enumerate}
    \item $\vec{AA}^+\vec{A}=\vec{A}$
    \item $\vec{A}^+\vec{AA}^+=\vec{A}^+$
    \item $\vec{AA}^+$ is symmetric
    \item $\vec{A}^+\vec{A}$ is symmetric
\end{enumerate}
\end{proof}

The general form of a conic is given by
\begin{align}
    \vec{x}^T\vec{Vx}+2\vec{u}^T\vec{x}+f=0 \label{eq:genConic}
\end{align}
for $y^2=4x$
\begin{align}
\vec{V}=\myvec{0 & 0 \\ 0 & 1},\ \vec{u}=\myvec{-2 \\0},\, f=0
\end{align}
For any given conic such that $\vec{V}$ is non-invertible, the points of tangency are given by
\begin{align}
    \vec{Vq}&=k\vec{n}-u \\
    \text{where},\, k& = \frac{\vec{p_1}^T\vec{u}}{\vec{p_1}^T\vec{n}},\, \vec{Vp_1}=0
\end{align}
Clearly
\begin{align}
    \vec{p_1} =&\myvec{1\\0} \hspace{2.0cm} \\
    \implies k  =&\frac{-2}{-1}=2
\end{align}
Using the obtained values
\begin{align}
    &\vec{Vq}=2\vec{n}-\vec{u}\\
    &\vec{Vq}=\myvec{-2\\2}-\myvec{-2\\0}=2\vec{e_2}
\end{align}
Using \eqref{eq:generalisedInverse} and \eqref{lemma:pseudoinverse}
\begin{align}
    \vec{q}&=2\vec{Ve_2}+(\vec{I-V^2})\vec{w}\\
    \vec{q}&=2\vec{e_2}+\lambda\vec{e_1} \label{eq:q_almostfound}
\end{align}
Using \eqref{eq:q_almostfound} and \eqref{eq:genConic}, we get
\begin{align}
    2\vec{q}^T\vec{e_2}-4\vec{e_1}^T\vec{q}=0
\end{align}
Evaluating this, we get
\begin{align}
    4-\lambda&=0 \\
    \implies \vec{q}&=\myvec{1\\2}
\end{align}


A plot for the line and parabola is given below
\begin{figure}[!ht]
    \centering
    \includegraphics[width=\columnwidth]{plot/figure_1}
    \caption{Plot of the line and parabola}
    \label{plot}
\end{figure}
\end{document}